\documentclass[xcolor={dvipsnames}]{beamer}
\usepackage[utf8]{inputenc}

\usetheme{Madrid}
\usecolortheme{default}
\setbeamertemplate{enumerate items}[default]
\setbeamercolor*{structure}{bg=white,fg=black}

% \setbeamercolor*{palette primary}{use=structure,fg=white,bg=structure.fg}
% \setbeamercolor*{palette secondary}{use=structure,fg=white,bg=structure.fg!75}
% \setbeamercolor*{palette tertiary}{use=structure,fg=white,bg=structure.fg!50!black}
% \setbeamercolor*{palette quaternary}{fg=white,bg=black}

% \setbeamercolor{section in toc}{fg=black,bg=white}
% \setbeamercolor{alerted text}{use=structure,fg=structure.fg!50!black!80!black}
% \setbeamercolor{frametitle}{bg=black,fg=white}

% \setbeamercolor{titlelike}{parent=palette primary,fg=structure.fg!50!black}
% \setbeamercolor{frametitle}{bg=gray!10!white,fg=black}

% \setbeamercolor*{titlelike}{parent=palette primary}

% \setbeamercolor{block title example}{bg=black,fg=white}

% \usepackage{times,url}

% \setbeamertemplate{footline}[frame number]
% \setbeamertemplate{footline}[frame number]{}

\setbeamertemplate{footline}{}

\setbeamertemplate{footline}
% {
%   \leavevmode%
%   \hbox{%
%   \begin{beamercolorbox}[wd=.333333\paperwidth,ht=2.25ex,dp=1ex,center]{author in head/foot}%
%     \usebeamerfont{author in head/foot}\insertsection
%   \end{beamercolorbox}%
%   \begin{beamercolorbox}[wd=.333333\paperwidth,ht=2.25ex,dp=1ex,center]{title in head/foot}%
%     \usebeamerfont{title in head/foot}\insertsubsection
%   \end{beamercolorbox}%
%   \begin{beamercolorbox}[wd=.333333\paperwidth,ht=2.25ex,dp=1ex,right]{date in head/foot}%
%     \usebeamerfont{date in head/foot}\insertshortdate{}\hspace*{2em}
%     \insertframenumber{} / \inserttotalframenumber\hspace*{2ex} 
%   \end{beamercolorbox}}%
%   \vskip0pt%
% }

%------------------------------------------------------------
%This block of code defines the information to appear in the
%Title page
\title[Learning Discriminative Representations to Interpret Image Recognition Models] %optional
{Learning Discriminative Representations to Interpret Image Recognition Models}

\subtitle{Thèse de Doctorat}

\author[Felipe Torres Figueroa] % (optional)
{Felipe Torres Figueroa}

\institute[VFU] % (optional)
{
  \'Ecole Centrale de Marseille
  \and
  Aix-Marseille Universit\'e
}

\date[VLC 2021] % (optional)
{Marseille, July 2024}

% 右下角使用logo的方式
\logo{\includegraphics[height=1cm]{overleaf-logo}} 

%End of title page configuration block
%------------------------------------------------------------



%------------------------------------------------------------
%The next block of commands puts the table of contents at the 
%beginning of each section and highlights the current section:

\AtBeginSection[]
{
  \begin{frame}
    \frametitle{Table of Contents}
    \tableofcontents[currentsection]
  \end{frame}
}
%------------------------------------------------------------


\begin{document}

%The next statement creates the title page.
\frame{\titlepage}


%---------------------------------------------------------
%This block of code is for the table of contents after
%the title page
\begin{frame}
\frametitle{Table of Contents}
\tableofcontents
\end{frame}
%---------------------------------------------------------


\section{First section}

%---------------------------------------------------------
%Changing visivility of the text
\begin{frame}
\frametitle{Sample frame title}
This is a text in second frame. For the sake of showing an example.

\begin{itemize}
    \item<1-> Text visible on slide 1
    \item<2-> Text visible on slide 2
    \item<3> Text visible on slides 3
    \item<4-> Text visible on slide 4
\end{itemize}
\end{frame}

%---------------------------------------------------------


%---------------------------------------------------------
%Example of the \pause command
\begin{frame}
In this slide \pause

the text will be partially visible \pause

And finally everything will be there
\end{frame}
%---------------------------------------------------------

\section{Second section}

%---------------------------------------------------------
%Highlighting text
\begin{frame}
\frametitle{Sample frame title}

In this slide, some important text will be
\alert{highlighted} because it's important.
Please, don't abuse it.

\begin{block}{Remark}
Sample text
\end{block}

\begin{alertblock}{Important theorem}
Sample text in red box
\end{alertblock}

\begin{examples}
Sample text in green box. The title of the block is ``Examples".
\end{examples}
\end{frame}
%---------------------------------------------------------


%---------------------------------------------------------
%Two columns
\begin{frame}
\frametitle{Two-column slide}

\begin{columns}

\column{0.5\textwidth}
This is a text in first column.
$$E=mc^2$$
\begin{itemize}
\item First item
\item Second item
\end{itemize}

\column{0.5\textwidth}
This text will be in the second column
and on a second tought this is a nice looking
layout in some cases.
\end{columns}
\end{frame}
%---------------------------------------------------------


\end{document}